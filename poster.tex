% Gemini theme
% https://github.com/anishathalye/gemini

\documentclass[final]{beamer}

% ====================
% Packages
% ====================

\usepackage[T1]{fontenc}
\usepackage{lmodern}
\usepackage[size=a4,scale=1.0]{beamerposter}
\usetheme{gemini}
\usecolortheme{gemini}
\usepackage{graphicx}
\usepackage{booktabs}
\usepackage{tikz}
\usepackage{pgfplots}
\pgfplotsset{compat=1.14}
\usepackage{anyfontsize}

% ====================
% Lengths
% ====================

% If you have N columns, choose \sepwidth and \colwidth such that
% (N+1)*\sepwidth + N*\colwidth = \paperwidth
\newlength{\sepwidth}
\newlength{\colwidth}
\setlength{\sepwidth}{0.025\paperwidth}
\setlength{\colwidth}{0.3\paperwidth}

\newcommand{\separatorcolumn}{\begin{column}{\sepwidth}\end{column}}

% ====================
% Title
% ====================

\title{Neural Representations of Consciousness in Human and Machine}
\author{Lucas Mahler}

% \author{Alyssa P. Hacker \inst{1} \and Ben Bitdiddle \inst{2} \and Lem E. Tweakit \inst{2}}
%
% \institute[shortinst]{\inst{1} Some Institute \samelineand \inst{2} Another Institute}

% ====================
% Footer (optional)
% ====================

% \footercontent{
%   \href{https://www.example.com}{https://www.example.com} \hfill
%   ABC Conference 2025, New York --- XYZ-1234 \hfill
%   \href{mailto:alyssa.p.hacker@example.com}{alyssa.p.hacker@example.com}}
% (can be left out to remove footer)

% ====================
% Logo (optional)
% ====================

% use this to include logos on the left and/or right side of the header:
% \logoright{\includegraphics[height=7cm]{logo1.pdf}}
% \logoleft{\includegraphics[height=7cm]{logo2.pdf}}

% ====================
% Body
% ====================

\begin{document}

\begin{frame}[t]
\begin{columns}[t]
\separatorcolumn

\begin{column}{\colwidth}

  \begin{block}{Introduction and Motivation}
Consciousness, with its subjective intricacies, is one of the most profound and persistent mysteries in contemporary scientific inquiry.
In this era, the question of consciousness emerges as an overarching challenge that remains unresolved.
Recent advances in artificial intelligence have amplified this challenge by prompting important questions about whether artificial neural systems can possess consciousness.
While a definitive theory remains elusive, a broad consensus prevails: a comprehensive understanding of the mind depends on a clear understanding of consciousness and its place within the natural order.
To advance our understanding of the mind, we must embark on an examination of both the nature of consciousness and its intricate relationship to non-conscious aspects of reality.

The "hard problem of consciousness" posed by Chalmers underscores the profound question: why do neural systems produce perceptions and emotions that are consciously experienced? An example that encapsulates this is the question of why the color red is experienced as "red".

Delving into the realm of consciousness not only satisfies our intellectual curiosity, but also proves indispensable in the field of medicine.
It aids in the diagnostic process, especially when faced with challenging conditions such as locked-in syndrome, or in making informed decisions during the administration of anesthesia.
In addition, the study of consciousness stimulates ethical considerations, particularly in situations involving end-of-life decisions for patients declared brain dead.

The emergence of machine consciousness adds a new dimension to the discourse.
It raises critical ethical concerns, especially in light of the widespread use of machine learning.
Addressing the potential for machine suffering, a concept termed "S-risk," emerges as an imperative aspect of this discussion.

The study of consciousness, both in human cognition and in artificial intelligence, enriches our understanding of the mind, improves medical practice, and guides us through the complicated terrain of AI ethics.
\end{block}

  \begin{block}{Backgound}
      \textbf{Global Neuronal Workspace Theory (GNWT):} Consciousness arises from the global broadcasting and amplification of information across interconnected networks, including prefrontal-parietal areas and high-level sensory cortical regions. Sensory information remains unconscious in encapsulated modules, with frontal-parietal networks supporting functions such as selective attention and working memory. Consciousness is triggered when stimuli are attended to, initiating global broadcasting across specialized submodules.

      \textbf{Integrated Information Theory (IIT):} Consciousness is the cause-effect power produced by maximally irreducible integrated information within specific neural architectures, ideally located in the posterior cortex. It's not just input-output processing, but an intrinsic ability of a network to influence itself. Conscious experience is the structure of cause and effect (integrated information), as opposed to the GNWT idea of a globally broadcast message.

      \textbf{Higher Order (HOT) and First Order (FOT) Theories of consciousness:} First-order theory holds that the activity of the sensory areas alone is sufficient for consciousness. Higher-Order Theory posits that a second, higher-order brain state must override first-order sensory activations for conscious experience to occur. This debate underscores the complex nature of consciousness and its underlying mechanisms.
  \end{block}

  \begin{alertblock}{Research Questions}
   \begin{itemize}
       \item Where and what are the anatomical footprints of consciousness in the brain and in artificial neural networks?
       \item Are these footprints located in a posterior cortical “hot zone” in both biological and artificial systems, as suggested by Integrated Information Theory (IIT)?
       \item How are conscious perceptions maintained over time, and is the underlying neural state similarly maintained in artificial systems?
       \item Is the conscious system initially ignited and then decays, remaining silent until a new ignition marks the onset of the next percept, in both biological and artificial systems?
    \end{itemize}
  \end{alertblock}

\end{column}

\separatorcolumn

\begin{column}{\colwidth}

  \begin{block}{Methods}
      \textbf{Brain encoding} is the process of translating external stimuli or cognitive processes into patterns of neural activity. This transformation facilitates the conversion of sensory input into understandable brain representations.

\textbf{Brain decoding}, also known as neural decoding or brain reading, interprets patterns of neural activity to reconstruct the information originally encoded in the brain. This approach gives us deeper insights into cognitive processes and how the mind works.

\textbf{Representational} similarity revolves around the use of neural activity patterns as representations of information. Using Representational Similarity Analysis (RSA), we quantitatively measure the similarity between these representations. This allows us to compare information encoded in the brain with information stored in artificial neural networks.

\textbf{Interpretability} is the ability to explain the decisions and predictions generated by machine learning models in a way that humans can understand. This promotes transparency, aids model refinement, and informs ethical considerations.

  \end{block}

  \begin{exampleblock}{Potential Porjects}

    \heading{Representations of Visual Consciousness}
In this potential project, we propose taking a pretrained video classifier and fine-tuning it using a dataset provided by Kronemer et al. This dataset is collected using a report-no-report paradigm, utilizing EEG and fMRI data acquisition methods. The primary objective is to conduct an in-depth analysis of representational similarity within the neural data, shedding light on the intricate representations of visual consciousness in the human brain.
    \heading{Implement Adversarial Collaboration Paradigm on 9T}
    Collect data following Melloni et al.'s protocols.
    In Experiment 1, we will assess critical predictions of the Global Neuronal Workspace (GNW) and Integrated Information Theory (IIT). Specifically, we will examine the decodability of clearly visible stimuli, regardless of their task relevance, and investigate the sustained activity of the physical substrate of consciousness during stimulus presentation.
    Experiment 2 focuses on measuring brain activity in response to salient visual stimuli reported as seen or unseen during a secondary task, manipulated through attention. This experiment primarily explores the neural mechanisms underlying conscious and unconscious processing. We will test multiple predictions from GNW and IIT, including the decodability of conscious content in prefrontal areas, the ignition of the global workspace, and patterns of connectivity within posterior cortical areas under various task conditions.
    \heading{Potential Project 3}
    ?

  \end{exampleblock}

\begin{block}{open questions}  
    \begin{itemize}
        \item Why 9.4T?
        \item Why at MPI kyb?
        \item Why fMRI?
        \item Why AI?
    \end{itemize}

\end{block}

\end{column}

\separatorcolumn

\begin{column}{\colwidth}

    \begin{block}{Impact}

The study of consciousness is of great importance, both in the realm of human cognition and in the context of artificial intelligence:

\heading{In Humans:}

\textbf{Curiosity:} Throughout history, consciousness has piqued the curiosity of philosophers and scientists alike. It remains one of the most enduring and profound subjects of human inquiry.
    Medical Implications: Understanding consciousness is central to the medical field. For example, the ability to determine whether an individual is conscious has significant medical implications. In cases such as comatose patients, it can distinguish conditions such as locked-in syndrome, in which bodily control is lost but perception persists to varying degrees. Additionally, in the context of anesthesia during surgery, understanding the nuances of consciousness becomes critical.

\textbf{End-of-life decisions:} In scenarios involving brain-dead patients, the question of consciousness is closely tied to ethical and medical decisions, including the possibility of withdrawing life support.

\heading{In Machines:}

\textbf{Suffering in machines:} Exploring consciousness in the context of artificial intelligence is also relevant. If machines were to possess consciousness, it raises the ethical concern that they might experience suffering. Given the vast scale of current machine learning deployments, the potential for compounded suffering becomes a significant consideration. This concept is encapsulated by the term "S-risk," which refers to the significant ethical risk associated with creating conscious machines.

\textbf{AI Safety:} Additionally, studying consciousness in AI has direct implications for AI safety, particularly concerning deceptive AI systems. Understanding how consciousness may emerge in AI can aid in the development of safeguards against unintended, harmful behavior in autonomous systems.

By delving into the study of consciousness in both humans and machines, we aim to unravel profound philosophical questions, improve medical practice, and address the ethical dimensions of AI advancement.
  \end{block}

  \begin{block}{References}

    \nocite{*}
    \footnotesize{\bibliographystyle{plain}\bibliography{poster}}

  \end{block}

\end{column}

\separatorcolumn
\end{columns}
\end{frame}

\end{document}
