% Gemini theme
% https://github.com/anishathalye/gemini

\documentclass[final]{beamer}

% ====================
% Packages
% ====================

\usepackage[T1]{fontenc}
\usepackage{lmodern}
\usepackage[size=a4,scale=1.0]{beamerposter}
\usetheme{gemini}
\usecolortheme{gemini}
\usepackage{graphicx}
\usepackage{booktabs}
\usepackage{tikz}
\usepackage{pgfplots}
\pgfplotsset{compat=1.14}
\usepackage{anyfontsize}

% ====================
% Lengths
% ====================

% If you have N columns, choose \sepwidth and \colwidth such that
% (N+1)*\sepwidth + N*\colwidth = \paperwidth
\newlength{\sepwidth}
\newlength{\colwidth}
\setlength{\sepwidth}{0.025\paperwidth}
\setlength{\colwidth}{0.3\paperwidth}

\newcommand{\separatorcolumn}{\begin{column}{\sepwidth}\end{column}}

% ====================
% Title
% ====================

\title{Neural Representations of Consciousness in Human and Machine}
\author{Lucas Mahler}

% \author{Alyssa P. Hacker \inst{1} \and Ben Bitdiddle \inst{2} \and Lem E. Tweakit \inst{2}}
%
% \institute[shortinst]{\inst{1} Some Institute \samelineand \inst{2} Another Institute}

% ====================
% Footer (optional)
% ====================

% \footercontent{
%   \href{https://www.example.com}{https://www.example.com} \hfill
%   ABC Conference 2025, New York --- XYZ-1234 \hfill
%   \href{mailto:alyssa.p.hacker@example.com}{alyssa.p.hacker@example.com}}
% (can be left out to remove footer)

% ====================
% Logo (optional)
% ====================

% use this to include logos on the left and/or right side of the header:
% \logoright{\includegraphics[height=7cm]{logo1.pdf}}
% \logoleft{\includegraphics[height=7cm]{logo2.pdf}}

% ====================
% Body
% ====================

\begin{document}

\begin{frame}[t]
\begin{columns}[t]
\separatorcolumn

\begin{column}{\colwidth}

  \begin{block}{Introduction and Motivation}
Consciousness, with its subjective intricacies, is one of the most profound and persistent mysteries in contemporary scientific inquiry.
In this era, the question of consciousness emerges as an overarching challenge that remains unresolved.
Recent advances in artificial intelligence have amplified this challenge by prompting important questions about whether artificial neural systems can possess consciousness.
While a definitive theory remains elusive, a broad consensus prevails: a comprehensive understanding of the mind depends on a clear understanding of consciousness and its place within the natural order.
To advance our understanding of the mind, we must embark on an examination of both the nature of consciousness and its intricate relationship to non-conscious aspects of reality.

The "hard problem of consciousness" posed by Chalmers underscores the profound question: why do neural systems produce perceptions and emotions that are consciously experienced? An example that encapsulates this is the question of why the color red is experienced as "red".

Delving into the realm of consciousness not only satisfies our intellectual curiosity, but also proves indispensable in the field of medicine.
It aids in the diagnostic process, especially when faced with challenging conditions such as locked-in syndrome, or in making informed decisions during the administration of anesthesia.
In addition, the study of consciousness stimulates ethical considerations, particularly in situations involving end-of-life decisions for patients declared brain dead.

The emergence of machine consciousness adds a new dimension to the discourse.
It raises critical ethical concerns, especially in light of the widespread use of machine learning.
Addressing the potential for machine suffering, a concept termed "S-risk," emerges as an imperative aspect of this discussion.

The study of consciousness, both in human cognition and in artificial intelligence, enriches our understanding of the mind, improves medical practice, and guides us through the complicated terrain of AI ethics.
\end{block}

  \begin{block}{Backgound}
      \textbf{Global Neuronal Workspace Theory (GNWT):} Consciousness arises from the global broadcasting and amplification of information across interconnected networks, including prefrontal-parietal areas and high-level sensory cortical regions. Sensory information remains unconscious in encapsulated modules, with frontal-parietal networks supporting functions such as selective attention and working memory. Consciousness is triggered when stimuli are attended to, initiating global broadcasting across specialized submodules.

      \textbf{Integrated Information Theory (IIT):} Consciousness is the cause-effect power produced by maximally irreducible integrated information within specific neural architectures, ideally located in the posterior cortex. It's not just input-output processing, but an intrinsic ability of a network to influence itself. Conscious experience is the structure of cause and effect (integrated information), as opposed to the GNWT idea of a globally broadcast message.

      \textbf{Higher Order (HOT) and First Order (FOT) Theories of consciousness:} First-order theory holds that the activity of the sensory areas alone is sufficient for consciousness. Higher-Order Theory posits that a second, higher-order brain state must override first-order sensory activations for conscious experience to occur. This debate underscores the complex nature of consciousness and its underlying mechanisms.
  \end{block}

  \begin{alertblock}{Research Questions}
   \begin{itemize}
       \item Where and what are the anatomical footprints of consciousness in the brain and in artificial neural networks?
       \item Are these footprints located in a posterior cortical “hot zone” in both biological and artificial systems, as suggested by Integrated Information Theory (IIT)?
       \item How are conscious perceptions maintained over time, and is the underlying neural state similarly maintained in artificial systems?
       \item Is the conscious system initially ignited and then decays, remaining silent until a new ignition marks the onset of the next percept, in both biological and artificial systems?
    \end{itemize}
  \end{alertblock}

\end{column}

\separatorcolumn

\begin{column}{\colwidth}

  \begin{block}{A block containing an enumerated list}
   \begin{enumerate}
      \item \textbf{Morbi mauris purus}, egestas at vehicula et, convallis
        accumsan orci. Orci varius natoque penatibus et magnis dis parturient
        montes, nascetur ridiculus mus.
      \item \textbf{Cras vehicula blandit urna ut maximus}. Aliquam blandit nec
        massa ac sollicitudin. Curabitur cursus, metus nec imperdiet bibendum,
        velit lectus faucibus dolor, quis gravida metus mauris gravida turpis.
      \item \textbf{Vestibulum et massa diam}. Phasellus fermentum augue non
        nulla accumsan, non rhoncus lectus condimentum.
    \end{enumerate}

  \end{block}

  \begin{block}{Fusce aliquam magna velit}
  \end{block}

  \begin{block}{Nam cursus consequat egestas}
  \end{block}

\end{column}

\separatorcolumn

\begin{column}{\colwidth}

  \begin{exampleblock}{A highlighted block containing some math}

    A different kind of highlighted block.

    $$
    \int_{-\infty}^{\infty} e^{-x^2}\,dx = \sqrt{\pi}
    $$

    Interdum et malesuada fames $\{1, 4, 9, \ldots\}$ ac ante ipsum primis in
    faucibus. Cras eleifend dolor eu nulla suscipit suscipit. Sed lobortis non
    felis id vulputate.

    \heading{A heading inside a block}

    Praesent consectetur mi $x^2 + y^2$ metus, nec vestibulum justo viverra
    nec. Proin eget nulla pretium, egestas magna aliquam, mollis neque. Vivamus
    dictum $\mathbf{u}^\intercal\mathbf{v}$ sagittis odio, vel porta erat
    congue sed. Maecenas ut dolor quis arcu auctor porttitor.

    \heading{Another heading inside a block}

    Sed augue erat, scelerisque a purus ultricies, placerat porttitor neque.
    Donec $P(y \mid x)$ fermentum consectetur $\nabla_x P(y \mid x)$ sapien
    sagittis egestas. Duis eget leo euismod nunc viverra imperdiet nec id
    justo.

  \end{exampleblock}

  \begin{block}{Nullam vel erat at velit convallis laoreet}

  \end{block}

  \begin{block}{References}

    \nocite{*}
    \footnotesize{\bibliographystyle{plain}\bibliography{poster}}

  \end{block}

\end{column}

\separatorcolumn
\end{columns}
\end{frame}

\end{document}
